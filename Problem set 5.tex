\documentclass[12pt]{article}%
\usepackage[paper=portrait,pagesize]{typearea}
\usepackage{amssymb}
\usepackage{amsfonts}
\usepackage{amsmath}
\usepackage{hyperref}
\usepackage{lscape}
\usepackage{comment}
\usepackage[flushleft]{threeparttable}
\usepackage{float}
\usepackage[nohead]{geometry}
\usepackage[singlespacing]{setspace}
\usepackage[paper=portrait,pagesize]{typearea}
\usepackage{amssymb}
\usepackage{amsfonts}
\usepackage{multicol}
\usepackage{amsmath}
\usepackage{hyperref}
\usepackage[nameinlink,noabbrev]{cleveref}
\usepackage{lscape}
\usepackage{float}
\usepackage[nohead]{geometry}
\usepackage[singlespacing]{setspace}
\usepackage[bottom]{footmisc}
\usepackage{indentfirst}
\usepackage{endnotes}
\usepackage{graphicx}%
\usepackage{afterpage}
\usepackage{subfig}
\usepackage{rotating}
\newcommand\tab[1][1cm]{\hspace*{#1}}
\DeclareMathOperator*{\Max}{Max}
\newcommand\numberthis{\addtocounter{equation}{1}\tag{\theequation}}
\def\dotfill#1{\cleaders\hbox to #1{.}\hfill}
\newcommand\dotline[2][.5em]{\leavevmode\hbox to #2{\dotfill{#1}\hfil}}
%\usepackage[backend=biber,style=alphabetic,sorting=ynt]{biblatex}
%\addbibresource{bibliocopulas.bib}
\usepackage[round,sort,comma,authoryear]{natbib}
\setcounter{MaxMatrixCols}{30}
\newtheorem{theorem}{Theorem}
\newtheorem{acknowledgement}{Acknowledgement}
\newtheorem{algorithm}[theorem]{Algorithm}
\newtheorem{axiom}[theorem]{Axiom}
\newtheorem{case}[theorem]{Case}
\newtheorem{claim}[theorem]{Claim}
\newtheorem{conclusion}[theorem]{Conclusion}
\newtheorem{condition}[theorem]{Condition}
\newtheorem{conjecture}[theorem]{Conjecture}
\newtheorem{corollary}[theorem]{Corollary}
\newtheorem{criterion}[theorem]{Criterion}
\newtheorem{definition}[theorem]{Definition}
\newtheorem{example}[theorem]{Example}
\newtheorem{exercise}[theorem]{Exercise}
\newtheorem{lemma}[theorem]{Lemma}
\newtheorem{notation}[theorem]{Notation}
\newtheorem{problem}[theorem]{Problem}
\newtheorem{proposition}{Proposition}
\newtheorem{remark}[theorem]{Remark}
\newtheorem{solution}[theorem]{Solution}
\newtheorem{summary}[theorem]{Summary}
\newenvironment{proof}[1][Proof]{\noindent\textbf{#1.} }{\ \rule{0.5em}{0.5em}}
\newcommand{\pd}[2]{\frac{\partial#1}{\partial#2}}
\makeatletter
\def\@biblabel#1{\hspace*{-\labelsep}}
\makeatother
\geometry{left=1in,right=1in,top=1.00in,bottom=1.0in}
%\renewcommand*\abstractname{Summary}

\begin{document}

\title{Fall 2019 - ECON 634 - Advance Macroeconomics - Problem Set 5}
\author{Elisa Taveras Pena\footnote{E-mail address: \href{mailto:etavera2@binghamton.edu}{etavera2@binghamton.edu}  }\\
Binghamton University}
\maketitle

\sloppy%avoids the breakage of words at the end of lines, by adjusting spaces between words inside the lines

\onehalfspacing

\begin{enumerate}
	\item Running the asked equation in matlab give me the following results:
	
	 \begin {table}[H]
	\begin{center}
			\caption {Estimation of OLS regression Using Card's data}
			{
				\begin{tabular}{l*{2}{c}}
					\hline
					&\multicolumn{1}{c}{Coefficient}&\multicolumn{1}{c}{St. Error}\\
					\hline\\
					[.25mm]
				Constant      &      4.9133***   &  (0.063121)   \\  
				Education         &    0.073807*** &   (0.0035336)  \\  
				Experience    &  0.039313*** &    (0.0021955)   \\ 
				smsa          &      0.16474***   &  (0.015692)   \\
				Black        &       -0.18822***  &  (0.017768)   \\
				South        &       -0.12905***  & (0.015229)   \\ 
				\hline
				\multicolumn{3}{l}{Root Mean Squared Error: 0.377}\\
				\hline	
				\end{tabular}
			}
	\end{center} 
	\end {table}
	
	\item The histogram of the variables are presented bellow:
	
	\begin{center}
		\includegraphics[width=0.7\linewidth]{Constact1}
	\end{center}
	\begin{center}
		\includegraphics[width=0.7\linewidth]{rmse1}
	\end{center}
	\begin{center}
		\includegraphics[width=0.7\linewidth]{South1}
	\end{center}
	\begin{center}
		\includegraphics[width=0.7\linewidth]{Black1}
	\end{center}
	\begin{center}
		\includegraphics[width=0.7\linewidth]{sms1}
	\end{center}
	\begin{center}
		\includegraphics[width=0.7\linewidth]{Exper1}
	\end{center}
	\begin{center}
		\includegraphics[width=0.7\linewidth]{Education1}
	\end{center}

The summary of the first two moments of the distributions can be seen in the following table:

	 \begin {table}[H]
\begin{center}
	\caption {Estimation of OLS regression Using Card's data}
	{
		\begin{tabular}{l*{2}{c}}
			\hline
			&\multicolumn{1}{c}{Mean}&\multicolumn{1}{c}{Variance}\\
			\hline\\
			[.25mm]
			Constant      &      4.9138   & 0.0040  \\  
			Education         &  0.0737 &  0.0000  \\  
			Experience    & 0.0393 &    0.0000   \\ 
			smsa          &    0.1658   &  0.0003   \\
			Black        &     -0.1894  &  0.0003 \\
			South        &      -0.1262 & 0.0002	  \\ 
			RMSE  			&   0.1425  & 0.0000\\
			\hline	
		\end{tabular}
	}
\end{center} 
\end {table}
	
	
	\item The value for education is given by 0.06, which using as prior give the following distribution:
	
	
		\begin{center}
		\includegraphics[width=0.7\linewidth]{Constact2}
	\end{center}
	\begin{center}
		\includegraphics[width=0.7\linewidth]{rmse2}
	\end{center}
	\begin{center}
		\includegraphics[width=0.7\linewidth]{South2}
	\end{center}
	\begin{center}
		\includegraphics[width=0.7\linewidth]{Black2}
	\end{center}
	\begin{center}
		\includegraphics[width=0.7\linewidth]{sms2}
	\end{center}
	\begin{center}
		\includegraphics[width=0.7\linewidth]{Exper2}
	\end{center}
	\begin{center}
		\includegraphics[width=0.7\linewidth]{Education2}
	\end{center}
	
	
	\item As we can see, although the distributions contain the actual value chosen on the OLS, 2. and 3. allows for not only have a mean value, which is pretty close to the OLS value, but also to have a distribution over the likely values of the coefficients. This allows for a more general understanding of the coefficient that a point estimation will gives up. 

\end{enumerate}

\strut

\onehalfspacing

\end{document}
